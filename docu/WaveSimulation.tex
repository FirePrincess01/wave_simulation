\documentclass[10pt,a4paper]{article}
\usepackage[utf8]{inputenc}
\usepackage{amsmath}
\usepackage{amsfonts}
\usepackage{amssymb}
\author{Sarah Steinbrecher}
\title{2D Wave Simulation}
\begin{document}

\section{Grundlagen}

\subsection{Approximative Ableitung zweiter Ordnung}

Rechter Differenzenquotient
\[ f'(x) \approx \frac{ f(x + h) - f(x) }{ h }  \]

Zentraler Differenzenquotient Zeiter Ordnung
\[ f''(x) \approx \frac{ f'( x + \frac{h}{2} ) - f'( x - \frac{h}{2} ) } { h }   \]

Zentraler Differenzenquotient 1
\[ f'( x + \frac{h}{2} ) \approx \frac{ f( x + h ) - f( x ) } { h }   \]

Zentraler Differenzenquotient 2
\[ f'( x - \frac{h}{2} ) \approx \frac{ f( x ) - f( x - h ) } { h }   \]

Zentraler Differenzenquotient Zeiter Ordnung
\[ f''(x) \approx \frac{ f( x + h ) - 2f(x) +  f( x - h )} { h^2 }   \]

\subsection{Laplace Operator}

\[ \Delta f = \Sigma^{n}_{k=1} \frac{\partial^2 f}{\partial x_k^2}   \]


\section{2D Wellen Simulation}

\subsection{Wellengleichung}

\[ \frac{\partial^2 u}{\partial t^2} -  \Delta_\mathbf{x} u(t, \mathbf{x}) = 0 \]

\subsection{Differentialdarstellung einer 1D Wellengleichung}

\[ z = u(t, x) \]

\[ \frac{\partial^2 u}{\partial t^2} = \frac{\partial^2 u}{\partial x^2} \]

Einsetzen der Approximativen Ableitung zweiter Ordnung (Verlet Methode)

\[ \frac{ u(t+\tau,x) - 2 u(t,x) + u(t-\tau,x) }{\tau^2} = \frac{ u(t,x+h) - 2 u(t,x) + u(t,x-h) }{h^2}  \]

Umgeformt

\[ u(t+\tau,x) = 2 u(t,x) + u(t-\tau,x) + \frac{\tau^2}{h^2}  ( u(t,x+h) - 2 u(t,x) + u(t,x-h) ) \]


\subsection{Differentialdarstellung einer 2D Wellengleichung}

\[ z = u(t, x, y) \]

\[ \frac{\partial^2 u}{\partial t^2} = \frac{\partial^2 u}{\partial x^2} + \frac{\partial^2 u}{\partial y^2} \]

Einsetzen der Approximativen Ableitung zweiter Ordnung (Verlet Methode)

\[ \frac{ u(t+\tau,x) - 2 u(t,x,y) + u(t-\tau,x,y) }{\tau^2} = \]
\[ \frac{ u(t,x+h,y) - 2 u(t,x,y) + u(t,x-h,y) }{h^2} + \frac{ u(t,x,y+h) - 2 u(t,x,y) + u(t,x,y-h) }{h^2}  \]

Umgeformt

\[ u(t+\tau,x,y) = \]
\[  2 u(t,x,y) - u(t-\tau,x,y) + \frac{\tau^2}{h^2}  [ u(t,x+h,y)  + u(t,x-h,y) + u(t,x,y+h) + u(t,x,y-h) - 4u(t,x,y) ] \]

Kernel


\[
u(t+\tau,x,y) = 2 u(t,x,y) - u(t-\tau,x,y) + \frac{\tau^2}{h^2} 
\begin{pmatrix}
  & 1 &  \\
1 & -4 & 1\\
  & 1 & \\
\end{pmatrix}
\]

\subsection{Startbedingung}

Gegeben $ u_0, v_0 $

\[ u_1 = u_0 + (u_0 - u_1) + \tau^2 u_0''  \]

\[ (u_0 - u_1) \approx \tau v_0 \]

\subsection{Randbedingungen}

Normalableitung ist 0 

n ... Einheitsnormalenvektor, welcher nach außen zeigt

n ist der Normalenvektor zu $\Gamma $

$\Gamma $ ist eine Linie, welche das Quadrat umgrenzt

$\Gamma ... Boundary$


\[ \frac{\partial u}{\partial \vec{n}} = 0 \]

\[ \forall x,y \in \Gamma \]

Approximativen Ableitung zweiter Ordnung am unteren Rand

\[ u(t+\tau,x,y) = \]
\[  2 u(t,x,y) - u(t-\tau,x,y) + \frac{\tau^2}{h^2}  [ u(t,x+h,y)  + u(t,x-h,y) + u(t,x,y+h) + 0 - 3u(t,x,y) ] \]

Kernel am unteren Rand

\[
u(t+\tau,x,y) = 2 u(t,x,y) - u(t-\tau,x,y) + \frac{\tau^2}{h^2} 
\begin{pmatrix}
  & 1 &  \\
1 & -3 & 1\\
  & 0 & \\
\end{pmatrix}
\]



\end{document}